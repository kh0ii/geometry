\begin{theorem}[Ptolemy]
	Cho tứ giác $ABCD$ nội tiếp đường tròn $(O)$. Khi đó ta có: 
	\[
		AB.CD + AD.BC = AC.BD
	\]
	\begin{center}	
		\includesvg{ptolemy}
	\end{center}

	\begin{proof}

		Trên $BD$ lấy $E$ sao cho $\angle DAE = \angle BAC$. Lại có $\angle ADE = \angle ACB$ nên $\triangle ADE \sim \triangle ACB \Rightarrow \frac{AD}{AC} = \frac{DE}{BC} \Rightarrow AD.BC = AC.DE \ (1)$.

		Do $\angle DAE = \angle CAB$ nên $\angle DAC = \angle EAB$ mà $\angle DCA = \angle EBA$ nên $\triangle ABE \sim \triangle ACD \Rightarrow \frac{AB}{BE} = \frac{AC}{CD} \Rightarrow AB.CD = AC.BE \ (2)$

		Từ $(1)$ và $(2) \Rightarrow AD.BC + AB.CD = AC.DE + AC.BE = AC.(DE + BE) = AC.BD$. 
	\end{proof}
\end{theorem}