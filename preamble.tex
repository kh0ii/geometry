%---------------------------
% Packages
%---------------------------
\usepackage[utf8]{vietnam}
\usepackage{amssymb, amsmath, latexsym, amsfonts, amsthm, mathrsfs, gensymb} % Standard packages that are nice to have.
\usepackage[inline]{asymptote}
\usepackage{svg}
\usepackage{amsrefs} % Allows for easy referencing and citations.
\usepackage{verbatim} % Needed for \begin{comment} \end{comment}.
\usepackage[text={6in,9in},centering]{geometry} % Defines the dimensions of the text body.
\usepackage[doublespacing]{setspace} % Makes the document double spaced.

% The following package makes prettier tables.  We're all about the bling!
\usepackage{booktabs}

% The units package provides nice, non-stacked fractions and better spacing
% for units.
\usepackage{units}

%--------Theorem Environments--------
\theoremstyle{definition}
\newtheorem{theorem}{Định lý}
\newtheorem{cor}[theorem]{Corollary}
\newtheorem{prop}[theorem]{Proposition}
\newtheorem{lemma}{Bổ đề}
\newtheorem{conj}[theorem]{Conjecture}
\newtheorem{quest}[theorem]{Question}
\newtheorem{claim}{Claim}

\theoremstyle{definition}
\newtheorem{defn}[theorem]{Định nghĩa}
\newtheorem{defns}[theorem]{Definitions}
\newtheorem{con}[theorem]{Construction}
\newtheorem{exmp}[theorem]{Example}
\newtheorem{jk}[theorem]{Joke}
\newtheorem{exmps}[theorem]{Examples}
\newtheorem{notn}[theorem]{Notation}
\newtheorem{notns}[theorem]{Notations}
\newtheorem{addm}[theorem]{Addendum}
\newtheorem{prob}{Bài}

\theoremstyle{remark}
\newtheorem{rem}[theorem]{Remark}
\newtheorem{ans}[theorem]{Answer}
\newtheorem{rems}[theorem]{Remarks}
\newtheorem{warn}[theorem]{Warning}
\newtheorem{sch}[theorem]{Scholium}

\newcommand{\csvg}[1]{\begin{center} \includesvg{#1} \end{center}}
\newcommand{\Mod}[1]{\ (\text{mod}\ #1)}
\newcommand{\R}{\mathbb{R}}
\newcommand{\N}{\mathbb{N}}
\newcommand{\Q}{\mathbb{Q}}
\newcommand{\F}{\mathbb{F}}
\newcommand{\Z}{\mathbb{Z}}
\renewcommand{\P}{\mathbb{P}}
\DeclareMathOperator{\Span}{Span}
\DeclareMathOperator{\val}{val}
\DeclareMathOperator{\comp}{comp}
\DeclareMathOperator{\im}{Im}
\DeclareMathOperator{\reg}{Reg}
\DeclareMathOperator{\odd}{Odd}
\DeclareMathOperator{\dist}{dist}
\DeclareMathOperator{\sbd}{sbd}
\DeclareMathOperator{\capac}{cap}