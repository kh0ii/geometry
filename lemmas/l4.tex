\begin{lemma}
	Cho $\triangle ABC$ nhọn có các đường cao $BE, CF$, $M$ là trung điểm $BC$. Khi đó $ME, MF$ và đường thẳng qua $A$ song song với $BC$ là tiếp tuyến của $(AEF)$.
\end{lemma}

\begin{center}
	\includesvg{l4}
\end{center}

\begin{proof}
	Gọi $H$ là trực tâm $\triangle ABC$.
	Có $\angle AFH = \angle AEH = 90 \degree$ nên $(AFE)$ có đường kính là $AH$.

	$AH \perp BC$ nên đường thẳng qua $A$ song song với $BC$ cũng vuông góc với $AH$. Vậy đường thẳng qua $A$ song song với $BC$ là tiếp tuyến của $(AEF)$.

	Có $BFEC$ là tứ giác nội tiếp do $\angle BFC = \angle BEC = 90 \degree$ nên $\angle FEB = \angle FCB$. $ME = MB = MC = MF$ nên $\triangle MEB, \triangle MEF$ cân tại $M$ nên ta có $\angle BEM = \angle MBE, \angle FME = \angle FEM$.

	Từ đó suy ra $\angle FEM = \angle FEB + \angle BEM = \angle FCB + \angle MBE = \angle CHE = \angle FAE$. Có $\angle FME = \angle FEM = \angle FAE$ nên $ME, MF$ là tiếp tuyến của $(AEF)$.
\end{proof}