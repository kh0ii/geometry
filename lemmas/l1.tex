\begin{lemma}[Incenter/Excenter Lemma]
Cho $\triangle ABC$ nội tiếp đường tròn tâm $I$. Tia $AI$ cắt $(ABC)$ tại điểm $L$ khác $A$. $I_a$ đối xứng với $I$ qua $L$. Khi đó
\begin{enumerate}
\item $I, B, C, I_a$ nằm trên đường tròn tâm $L$ đường kính $I I_a$ hay $LI = LB = LC = L I_a$.
\item Tia $B I_a$ và $C I_a$ là phân giác góc ngoài của $\triangle ABC$.
\end{enumerate}

\begin{center}
\includesvg{l1}
\end{center}

\end{lemma}

\begin{proof}

Đầu tiên ta sẽ chứng minh $LB = LI$. Có 
$$
\angle LBI = \angle LBC + \angle CBI = \angle LAC + \angle IBA = \angle IAB + \angle IBA = \angle LIB.
$$
Vậy $\triangle LBI$ cân tại $L$ nên $LB = LI$. Tương tự, ta có $LC = LI$. Vì $LI = LB = LC$ nên $L$ là tâm $(BIC)$. Mà $L$ là trung điểm $I I_a$ nên $I I_a$ là đường kính của $(BIC)$.

$I I_a$ là đường kính $(BIC)$ nên $\angle I_a BI = 90 \degree$ hay $B I_a \perp BI$ mà $BI$ là phân giác $\angle ABC$ nên $B I_a$ là phân giác của góc ngoài $\triangle ABC$ tại $B$. Tương tự có $C I_a$ là phân giác góc ngoài $\triangle ABC$ tại $C$.

\end{proof}