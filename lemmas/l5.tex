\begin{lemma}[Bổ đề hình thang]
	Trong hình thang hai đáy không bằng nhau, giao điểm của hai đường thẳng chứa hai cạnh bên, giao điểm của hai đường chéo và trung điểm của hai đáy cùng nằm trên một đường thẳng.
\end{lemma}

\begin{center}
\includesvg{l5}
\end{center}

\begin{proof}
	Cho hình thang $ABCD$, $M$ là giao của hai cạnh bên $AD$ và $CB$, $E$ là trung điểm $AC$, $F$ là trung điểm $BD$, $O$ là giao của $AC$ và $BD$.

	$\triangle ABO \sim \triangle CDO \ (g.g)$ nên $\frac{AB}{CD} = \frac{AO}{CO} \Rightarrow \frac{2AE}{2CF} = \frac{AO}{CO} \Rightarrow \frac{AE}{CF} = \frac{AO}{CO}$. Lại có $\angle EAO = \angle FCO$ nên $\triangle AEO \sim \triangle CFO \Rightarrow \angle AOE = \angle COF$ nên $E, O, F$ thẳng hàng $(1)$.

	Có $\triangle AEM \sim \triangle DFM \ (g.g) \Rightarrow \angle AME = \angle DMF$ nên $M, E, F$ thẳng hàng $(2)$. 

	Từ (1) và (2) ta có điều phải chứng minh.
\end{proof}