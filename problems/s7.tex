\begin{prob}[Iran TST 2011/1]
	Cho tam giác $ABC$ nhọn có $\angle B$ lớn hơn $\angle C$. Gọi $M$ là trung điểm $BC$ và $E, F$ là chân đường cao hạ lần lượt từ $B$ và $C$. Gọi $K, L$ lần lượt là trung điểm của $ME$ và $MF$ và gọi $T$ là điểm nằm trên đường thẳng $KL$ sao cho $TA \parallel BC$. Chứng minh rằng $TA = TM$.
\end{prob}

\begin{center}
\includesvg{s7}
\end{center}

Từ \textbf{Bổ đề 5} ta có $ME, MF, AT$ là các tiếp tuyến của $(AEF)$. Gọi $\omega$ là đường tròn tâm $M$ bán kính $0$. $K$ là trung điểm $ME$ nên $KE^2 = KM^2$, $K$ là điểm có phương tích tới $(AEF)$ và $\omega$ bằng nhau nên $K$ thuộc trục đẳng phương của $(AEF)$ và $\omega$. Chứng minh tương tự ta cũng có $L$ thuộc trục đẳng phương của $(AEF)$ và $\omega$ nên đường thẳng $KL$ chính là trục đẳng phương của hai đường tròn này. Có $T \in KL$, $TA$ là tiếp tuyến của $(AEF)$ nên $TA^2 = TM^2$ suy ra $TA = TM$.