\begin{prob}[IMO Shortlist 2010/G1]
	Cho tam giác $ABC$ nhọn với $D, E, F$ lần lượt là chân đường cao trên $BC, CA, AB$. Một trong hai giao điểm của đường thằng $EF$ với đường tròn ngoại tiếp tam giác $ABC$ là $P$. Hai đường thẳng $BP$ và $DF$ cắt nhau tại $Q$. Chứng minh $AP = AQ$.
\end{prob}

\begin{center}
\includesvg{s1_1} \ \ \ \ \ \ \ \ \ \ \includesvg{s1_2}
\end{center}

\textit{Trường hợp 1:} $P$ thuộc tia $EF$.

Ta thấy $\angle APQ = \angle ACB$ vì tứ giác $APBC$ nội tiếp.

Dễ chứng minh tứ giác $AFDC, BFEC$ nội tiếp, từ đó ta có $\angle AFQ = \angle ACB$ và $\angle AFE = \angle ACB$.

Vậy $\angle APQ = \angle AFQ$ suy ra tứ giác $APFQ$ nội tiếp. Do đó $\angle AFE = \angle AQP$ mà $\angle AFE = \angle APQ = \angle ACB$ nên $\angle AQP = \angle APQ$ suy ra $\triangle APQ$ cân tại $A$. Vì vậy $AP = AQ$, ta có điều phải chứng minh.

\textit{Trường hợp 2:} $P$ thuộc tia $FE$.

Có $\angle APQ = \angle ACB = \angle BFD = 180 \degree - \angle AFQ$ nên tứ giác $APQF$ nội tiếp.

Từ đó có $\angle AQP = \angle AFP = \angle ACB$ mà $\angle APQ = \angle ACB$ nên $\angle APQ = \angle AQP$. Vậy $\triangle APQ$ cân tại $A$ nên $AP = AQ$.