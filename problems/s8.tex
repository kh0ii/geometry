\begin{prob}[IMO 2000/1]
	Hai đường tròn $G_1$ và $G_2$ cắt nhau tại hai điểm $M$ và $N$. $AB$ là tiếp tuyến của hai đường tròn lần lượt tại $A$ và $B$ sao cho $M$ nằm gần $AB$ hơn so với $N$. Gọi $CD$ là đường thẳng qua $M$ song song với $AB$, với $C$ thuộc $G_1$ và $D$ thuộc $G_2$. $AC$ cắt $BD$ tại $E$; $AN$ cắt $CD$ tại $P$; $BN$ cắt $CD$ tại $Q$. Chứng minh rẳng $EP = EQ$.
\end{prob}

\begin{center}
\includesvg{s8}
\end{center}

Gọi $MN$ cắt $AB$ tại $I$, khi đó $IA^2 = IB^2$ suy ra $IA = IB$ do $I$ thuộc trục đẳng phương $MN$ của hai đường tròn $G_1$ và $G_2$. Theo \textbf{Bổ đề 5} ta cũng có $M$ là trung điểm $PQ$.

Do $AB$ là tiếp tuyến của $G_1$ và $CD \parallel AB$ nên $\angle MAB = \angle MCA = \angle EAB$. Tương tự ta có $\angle MBA = \angle EBA$. Vì vậy $\triangle EAB = \triangle MAB \ (c.g.c)$ suy ra $AE = AM, BE = BM$ hay $AB$ là trung trực $ME$. $AB \perp EM$, $AB \parallel CD$ suy ra $EM \perp CD$ mà $M$ là trung điểm $PQ$ suy ra tam giác $EPQ$ cân tại $E$, suy ra $EP = EQ$.