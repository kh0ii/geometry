
\begin{prob}[CGMO 2012/5]
	Cho $\triangle ABC$. Đường tròn nội tiếp $\triangle ABC$ có tâm là $I$ và tiếp xúc với $AB, AC$ lần lượt tại $D, E$. Gọi $O$ là tâm đường tròn ngoại tiếp $\triangle BCI$.
\end{prob}

\begin{center}	
\includesvg{s2}
\end{center}	

Theo \textbf{Bổ đề 1}, $O \in AI$. 

Có $AD, AE$ là hai tiếp tuyến đến $(I)$ nên $AI$ là trung trực $DE$ suy ra $AD = AE$. 

Khi đó $\triangle ADO = \triangle AEO \ \text{(c - g - c)}$ nên $\angle ADO = \angle AEO$, từ đây dễ dàng thấy được $\angle ODB = \angle OEC$.